%\documentclass[a4paper]{article}
%% Language and font encodings
\documentclass[twocolumn,aps,prl]{revtex4-1}
\usepackage[utf8]{inputenc}
\usepackage[spanish, es-tabla]{babel}
\usepackage[T1]{fontenc}
\usepackage{amsmath}
\usepackage{amssymb}
\usepackage{siunitx}
\usepackage{multirow}
\usepackage{float}
\usepackage{enumitem} % enumerar

\sisetup{math-micro=\text{µ},text-micro=µ}

\usepackage[toc,page]{appendix}

%% Sets page size and margins
\usepackage[a4paper,top=1.5cm,bottom=2cm,left=1.7cm,right=1.7cm,marginparwidth=1.75cm]{geometry}

%% Sets caption text size(its bigger than text)
\usepackage{caption}
\captionsetup[figure]{font=small}
\usepackage{subcaption}

%% Useful packages
\usepackage{svg}
\usepackage{epstopdf}
\usepackage{amsmath}
\usepackage{graphicx}
% \usepackage[showframe]{geometry}% http://ctan.org/pkg/geometry
%\usepackage[colorinlistoftodos]{todonotes}
\usepackage[colorlinks=true, allcolors=blue]{hyperref}

\newcommand{\nstar}{n^*} 
\newcommand{\Nstar}{N^*} 
\newcommand{\talf}{\frac{\alpha - 1}{\alpha \beta - 1} } 
\newcommand{\tbet}{\frac{\beta  - 1}{\alpha \beta - 1} }  

%%%%%%%%%%%%%%%%%%%%%%%%%%%%%%%%%%%%%%%%%%%%%%%%%%%%%%
%%%%%%%%%%%%%%%%%%%%%%%%%%%%%%%%%%%%%%%%%%%%%%%%%%%%%%
%%%%%%%%%%%%%%%%%%%%%%%%%%%%%%%%%%%%%%%%%%%%%%%%%%%%%%
%%%%%%%%%%%%%%%%%%%%%%%%%%%%%%%%%%%%%%%%%%%%%%%%%%%%%%
%%%%%%%%%%%%%%%%%%%%%%%%%%%%%%%%%%%%%%%%%%%%%%%%%%%%%%

\begin{document}

% ██   ██ ███████  █████  ██████
% ██   ██ ██      ██   ██ ██   ██
% ███████ █████   ███████ ██   ██
% ██   ██ ██      ██   ██ ██   ██
% ██   ██ ███████ ██   ██ ██████

\title{Practico 1}
\author{M. G. Aramayo}
\affiliation{Matematica de sistemas biologicos, Instituto Balseiro}

% \begin{abstract}
% Mete acá las conclusiones.
% \end{abstract}

\maketitle


% ███████╗██╗  ██╗ ██╗
% ██╔════╝╚██╗██╔╝███║
% █████╗   ╚███╔╝ ╚██║
% ██╔══╝   ██╔██╗  ██║
% ███████╗██╔╝ ██╗ ██║
% ╚══════╝╚═╝  ╚═╝ ╚═╝

\section{Resolucion Ej 1:}


% 
% ███████╗██╗  ██╗    ██████╗  
% ██╔════╝╚██╗██╔╝    ╚════██╗
% █████╗   ╚███╔╝      █████╔╝
% ██╔══╝   ██╔██╗     ██╔═══╝ 
% ███████╗██╔╝ ██╗    ███████╗
% ╚══════╝╚═╝  ╚═╝    ╚══════╝

\section{Resolucion Ej 2:}

% 
% ███████╗██╗  ██╗    ██████╗     
% ██╔════╝╚██╗██╔╝    ╚════██╗    
% █████╗   ╚███╔╝      █████╔╝    
% ██╔══╝   ██╔██╗      ╚═══██╗    
% ███████╗██╔╝ ██╗    ██████╔╝    
% ╚══════╝╚═╝  ╚═╝    ╚═════╝     
%                                 
% 

\section{Resolucion Ej 3:}

Para el siguiente sistema de competencia ciclica:
$$
\begin{aligned}
\frac{d n_{1}}{d t}&=n_{1}\left(1-n_{1}-\alpha n_{2}-\beta n_{3}\right) \\
\frac{d n_{2}}{d t}&=n_{2}\left(1-\beta n_{1}-n_{2}-\alpha n_{3}\right) \\
\frac{d n_{3}}{d t}&=n_{3}\left(1-\alpha n_{1}-\beta n_{2}-n_{3}\right)
\end{aligned}
$$
con $0<\beta<1<\alpha$ y $\alpha+\beta>2$. Pueden obtenerse los  equilibrios viendo todos los puntos que cumplan simultaneamente: 

$$
\begin{aligned}
    f_1(\nstar_1,\nstar_2,\nstar_3) = 0\\ 
    f_2(\nstar_1,\nstar_2,\nstar_3) = 0\\ 
    f_3(\nstar_1,\nstar_2,\nstar_3) = 0   
\end{aligned}
$$

Son 8 puntos de equilibrios $P_j = (\nstar_{1,j},\nstar_{2,j},\nstar_{3,j}), j= 1, 3, ..., 8$.

$$
\begin{aligned}
    P_1 &= (0, 0, 0) \\ 
    P_2 &= (1, 0, 0) \\ 
    P_3 &= (0, 1, 0) \\ 
    P_4 &= (0, 0, 1) \\  
    P_5 &= \frac{1}{\alpha \beta - 1} 
                       (\alpha - 1, \beta - 1 , 0) \\ 
    P_6 &= \frac{1}{\alpha \beta - 1} 
                       (0         , \alpha - 1, \beta - 1) \\ 
    P_7 &= \frac{1}{\alpha \beta - 1} 
                       (\beta - 1 , 0         , \alpha - 1) \\ 
    P_8 &= \frac{1}{\alpha + \beta + 1}(1, 1, 1) 
\end{aligned}
$$

Analizamos el la matriz jacobiana de este sistema:

% $$
% \mathbf{J} = 
% \begin{bmatrix}
%     1 - 2 x - \alpha y - \beta z & - \alpha x & - \beta x \\
%     - \beta y & 1 - 2 y - \beta x - \alpha z & - \alpha y \\
%     - \alpha z & - \beta z & 1 - \alpha x - \beta y - 2 z
% \end{bmatrix}
% $$

$$
\mathbf{J}_5 = 
\begin{bmatrix}
    -\talf  & - \alpha (\talf)& - \beta (\talf)\\
    - \beta (\tbet) & -\tbet & - \alpha (\tbet) \\
    0 & 0 & 1 - \alpha (\talf) - \beta (\tbet)
\end{bmatrix}
$$

$$
\mathbf{J}_6 = 
\begin{bmatrix}
    1 - \alpha (\talf) - \beta (\tbet) & 0 & 0 \\
    - \beta (\talf) & -\talf & - \alpha (\talf) \\
    - \alpha (\tbet) & - \beta (\tbet) & \tbet
\end{bmatrix}
$$

$$
\mathbf{J}_7 = 
\begin{bmatrix}
    -\tbet & - \alpha (\tbet) & - \beta (\tbet) \\
    0 & 1 - \beta (\tbet) - \alpha (\talf) & 0 \\
    - \alpha (\talf) & - \beta (\talf) & \talf
\end{bmatrix}
$$


$$
\mathbf{J}_1 = 
\begin{bmatrix}
    1  & 0 & 0 \\
    0 & 1 & 0 \\
    0 & 0 & 1
\end{bmatrix}
\quad 
\mathbf{J}_8 = 
\frac{1}{1+\alpha+\beta}\begin{bmatrix}
    -1 & -\alpha & -\beta \\
    -\beta & -1 & -\alpha \\
    -\alpha & -\beta & -1
\end{bmatrix}
$$

$$
\mathbf{J}_2 = 
\begin{bmatrix}
    -1 & - \alpha & - \beta \\
    0 & 1-\beta & 0 \\
    0 & 0 & 1- \alpha
\end{bmatrix}
$$

$$
\mathbf{J}_3 = 
\begin{bmatrix}
    1 - \alpha  & 0 & 0 \\
    - \beta & -1 & - \alpha \\
    0 & 0 & 1  - \beta
\end{bmatrix}
,
\mathbf{J}_4 = 
\begin{bmatrix}
    1 - \beta  & 0 & 0 \\
    0 & 1 -\alpha  & 0 \\
    - \alpha  & -\beta & -1
\end{bmatrix}
$$



El origen es equilibrio (una fuente) y los versores $(1,0,0)$ etc. son puntos de ensilladura. Hay otros 3 equilibrios con dos poblaciones finitas y una nula. Finalmente, existe un equilibrio interior al octante $\mathbb{R}_{+}^{3}$, dado por:
$$
x_{1}^{*}=x_{2}^{*}=x_{3}^{*}=\frac{1}{1+\alpha+\beta} .
$$
Este equilibrio de coexistencia es una ensilladura, lo cual se demuestra fácilmente porque los autovalores son muy sencillos. La matriz del sistema linealizado es "circulante":
$$
\frac{1}{1+\alpha+\beta}\left(\begin{array}{ccc}
-1 & -\alpha & -\beta \\
-\beta & -1 & -\alpha \\
-\alpha & -\beta & -1
\end{array}\right)
$$
con lo que sus autovalores son combinaciones de las raíces cúbicas de la unidad:
$$
\lambda_{k}=\sum_{j=0}^{n-1} c_{j} \gamma_{j}^{k}, \quad k=0,1, \ldots, n-1
$$
con $c_{j}$ los elementos de la matriz y $\gamma_{j}$ las raíces de la unidad, $\gamma_{j}=\exp (2 \pi i / n)$, en general. Así que:
$$
\begin{array}{c}
\lambda_{0}=-1, \text { con autovector }(1,1,1) \\
\lambda_{1}=\lambda_{2}^{*}=\frac{1}{1+\alpha+\beta}\left(-1-\alpha e^{2 x i / 3}-\beta e^{4 \pi i / 3}\right),
\end{array}
$$
que satisfacen:
$$
\operatorname{Re}\left(\lambda_{1}\right)=\operatorname{Re}\left(\lambda_{2}\right)=\frac{1}{1+\alpha+\beta}\left(-1+\frac{\overbrace{\alpha+\beta}^{>2}}{2}\right)>0
$$

% 
% ███████╗██╗  ██╗    ██╗  ██╗
% ██╔════╝╚██╗██╔╝    ██║  ██║
% █████╗   ╚███╔╝     ███████║
% ██╔══╝   ██╔██╗     ╚════██║
% ███████╗██╔╝ ██╗         ██║
% ╚══════╝╚═╝  ╚═╝         ╚═╝
%                             
% 

\section{Resolucion Ej 4:}


% 
% ███████╗██╗  ██╗    ███████╗
% ██╔════╝╚██╗██╔╝    ██╔════╝
% █████╗   ╚███╔╝     ███████╗
% ██╔══╝   ██╔██╗     ╚════██║
% ███████╗██╔╝ ██╗    ███████║
% ╚══════╝╚═╝  ╚═╝    ╚══════╝
%                             
% 

\section{Resolucion Ej 5:}

% 
% ███████╗██╗  ██╗     ██████╗ 
% ██╔════╝╚██╗██╔╝    ██╔════╝ 
% █████╗   ╚███╔╝     ███████╗ 
% ██╔══╝   ██╔██╗     ██╔═══██╗
% ███████╗██╔╝ ██╗    ╚██████╔╝
% ╚══════╝╚═╝  ╚═╝     ╚═════╝ 
%                              
% 

\section{Resolucion Ej 6:}

\bibliography{sample}

\end{document}

% ███    ██  ██████  ████████  █████  ███████
% ████   ██ ██    ██    ██    ██   ██ ██
% ██ ██  ██ ██    ██    ██    ███████ ███████
% ██  ██ ██ ██    ██    ██    ██   ██      ██
% ██   ████  ██████     ██    ██   ██ ███████


% ████████ ██    ██ ██████   ██████  ███████
%    ██     ██  ██  ██   ██ ██    ██ ██
%    ██      ████   ██████  ██    ██ ███████
%    ██       ██    ██      ██    ██      ██
%    ██       ██    ██       ██████  ███████

% A pesar de mis multiples y cultas intervenciones seguis escribiendo así, usa ctrl f y resolvelo.

% atomico
% volumen
% parametro
% mantenia
% dielectrico
% perdida
% ferroelectrico
% difractograma
% difractometro
% minimo
% maximo
% tension
% conversion
% aislacion
% medicion
% resolucion
% funcion
% transicion
% correccion
% activacion
% correlacion
% tipico X
% habia  X
% agrego X
